\documentclass[12pt, oneside]{book}
\title{Wrap Your Mind Around Vim 8.0}
\author{John Filippone}
\usepackage{enumitem}

\begin{document}
\maketitle
\tableofcontents

\chapter{Preamble}
\section{Purpose}
The object of this book is to provide the key knowledge and frame of mind necessary for the reader to achieve a highly efficient workflow with Vim and to gain command over its vast capabilities.

\section{Vim at a Glance}
As stated on vim.org: Vim is a highly configurable text editor built to make creating and changing any kind of text very efficient. It is included as ``vi" with most UNIX systems and with Apple OS X.

\vspace{5mm}
Vi was developed for UNIX and first released in 1978.  ``Vim" is short for Vi Improved.  Vim was initially modeled after vi and since the first release of Vim in 1991, Vim development has been highly active with the most recent release to date of this book being Vim 8.0 in 2016.

\section{Important Historical Perspective}
When Vim was created, it was made by Bram for Bram; meaning, the engineer who developed Vim did so for his personal use.  He developed the features that he personally wanted to use.  As years went on
and Bram continued developing, the community asked for other features that Bram did not necessarily care about.  He implemented them anyway as long as he felt enough people would enjoy having them.
After decades of development, Vim has become a vast collection of features.  There are too many for any one person to really need or want or use them all.  The expansive collection of features is
meant to accommodate a diverse set of users where each user only needs and/or wants a small subset of all features to achieve their ideal workflow.  It is for this reason that this book focuses on
understanding Vim rather than a complete learning of the tool.  Getting the maximum utility from Vim on your particular workflow does not require you to know how to use all the features but rather
(usually) only 5-10 percent of them.

\section{This Book's Unique Approach}
This is not a 'HOW to use Vim' book.  Instead, this book focuses on teaching WHAT Vim can do.  The approach of this book is based on the premise that learning HOW to use a particular feature in Vim is
easy; the hard part is figuring out the WHAT feature you want to learn.  Because Vim is such an extensively customizable tool, the HOW of a feature is not particularly important when it comes to
achieving efficient workflow.  If the default invocation of a feature does not work for you, there is almost definitely a way to change the invocation to something that does work.  Any how-to elements
of this book will be minimal for the sake of brevity.  This book will instead into detail on the WHAT is possible in Vim and WHAT the proper terminology is.  Once you know what the thing you want to
do is and what it is called, all it takes some internet searching and some practice.  If you try to dive right into learning all of Vim without this type of understanding you will waste a lot of time.
You will never use most of what you learn.  It is most effective to understand Vim first and understand what it has to offer your personal workflow, then only learn those parts of Vim.  After reading
this book you will have effectively wrapped your mind around Vim.  You will have a strong understanding of the majority of the moving parts of Vim and the functionality that they offer.  Armed with that understanding, you can quickly determine the set of features you need for your particular workflow.  More importantly, the next time your workflow changes, your knowledge of Vim will allow you to quickly assess whether or not Vim is the tool for the job (more than likely it is) and you will know what to search for when learning how to invoke the functionality you want.

\section{How Most People Learn Vim}
Most people are thrust into Vi and/or Vim and are forced to frantically learn the basic editing commands.  Even very smart users get stuck in Vim because they do not understand the capabilities of the
tool or the technical colloquialisms associated with it.  When struggling with a problem, the new user has two main issues.  They do not know that Vim is capable of doing the thing they want or they
know what they want and suspect that Vim can do it but have no idea what the feature is called.  Unfortunately, most books for Vim are recipe books.  They provide tutorial on specific features or
describe great ways to solve common problems with Vim.  These are great for picking up tricks but you will brush through hundreds of pages before learning the key set of Vim features that work best
for you and the task you are applying them to.  Most will try these 'HOW to' approaches, then quickly give up on them and default to being satisfied with a cumbersome workflow.  The internet can be
very helpful if you know what you are looking for as long as you don't fall into a wormhole of mystifying power-user language.  The weird dialog about buffers and Ex-mode and autocommands may be easy
enough for you to sift through but the real trouble is the former.  When you start using Vim, you have no idea what you need to learn!  Most have a vague idea of what functionality they want, but do not know the proper Vim terminology, making internet searching for a how-to very difficult.  Most users will pick up a trick here and there over the years from their coworkers or on the internet after getting fed up with a difficult workflow, but never fully feel comfortable working with Vim.

\section{How You Ought To Learn Vim}
\begin{enumerate}
\item Understand what Vim has to offer. (Wrap Your Mind Around Vim)
\item Pick a subset of key features you need to learn in order to have an efficient workflow for your specific task.
\item Learn each of these features through the Vim help files, books, and/or the internet.
\item Practice using them enough to commit them to muscle memory for a truly efficient workflow.
\item When you change your workflow, use your understanding from step one and revert to step 2.
\end{enumerate}
This is an efficient method of learning Vim.  Only learn the features you actually care to use.  Know what the tool can do so that later on down the road when you need some other functionality, you
know what to look for.  This book aids you with the most important step in the process: understanding Vim.

\section{How to Use This Book}
You can read this straight through or you can skip to the parts you really care about.  It will be valuable to at least skim all of the material so that you come away with an understanding of the full
scope of Vim capabilities.  If you are mystified by the terminology in this book, refer to the Vim Terminology chapter.

\chapter{Vim Terminology}
These are words that you may come across in this book or on the internet in a Vim context.  Many of these can be very mystifying if you don't understand their special Vim meaning.  Throughout this
book, these words are in bold when used in the context detailed in this section.  For example if this book uses the word "buffer" but it is not bold, think in terms of a normal dictionary definition.
If it is bold, use the meaning detailed in this section.  Each of these terms can also be searched in the vim help files by using the command :help <term>

\begin{itemize}[leftmargin=*, label={}]
  \item \textbf{abbreviations} An abbreviation is a mapping between two strings.  Once you create the abbreviation, when you type the abbreviation it will automatically be replaced (while you are
    typing) by the string you mapped it to.  Abbreviations apply in Insert mode, Replace mode and Command-line mode.  This can be used to save typing for long words that you use often.  And you can
    also use it to automatically correct obvious spelling errors.
  \item \textbf{alternate-file} (A.K.A alternate buffer) The alternate file is the name of the file that was edited previously.  When you open a new file in a window, the name of that file is the
    current file.  If there was already a current file in the window that the new file was opened in, then the name of that file becomes the alternate file.  If you like using more than one buffer in
    the same window, marking the previously edited file as the alternate file allows for useful commands that help you navigate between buffers.  However, if you do not ever edit more than one file
    per window, you will probably never interface with the alternate file mechanism.
  \item \textbf{argument-list} This refers to the arguments you provide on the command line when starting vim.  Do not confuse this with the buffer-list.  The argument is global for all windows in the
    vim session.
  \item \textbf{autocommand} Autocommands are a Vim mechanism that allows you to program vim to automatically execute some vim script when an event happens.  There is a large variety of supported
    events and you are only limited by what Vim script can accomplish.  One common application is to create an autocommand that will delete all trailing white space in the file when you
    write the file.
  \item \textbf{ctags} One of many external programs that generates tag files.  Tag files created by ctags, or a variety of other compatible programs, allow Vim to have awareness of where functions,
    classes, variables and other identifiers exist and are defined.  This opens the door to a lot of very powerful navigation.
  \item \textbf{current-file} (A.K.A current buffer) The current file refers the name of the file that is currently being edited.  Each window has a current file.  If you like using more than one
    buffer in the same window, the concept of a current file becomes meaningful when navigating between buffers.  However, if you do not ever edit more than one file per window, you will probably never
    interface with current file because you will always be editing the current file.
  \item \textbf{dotfiles} This refers to hidden configuration files in general.  Vim uses a number of dot files.  They are called dotfiles because their names start with a dot (i.e. .vimrc, .gvimrc,
    .exrc, etc.).  In unix, starting a filename with a dot makes it a hidden file.  Vim supports a variety of dotfiles but the most commonly used is the .vimrc.  Simply create a file in your home
    directory called .vimrc and any vim script you write in that file will be executed every time you open Vim.
  \item \textbf{events} In Vim context, events almost certainly refers to autocommand events.  An event simply consists of a name and a definition of the action that triggers the event.  Search autocommand-events in the vim help files to find a complete list of supported autocommand events.
  \item \textbf{ex editor} The ex editor is the predecessor to vi, which is the editor that Vim was initially modeled after.  The ex editor is a line editor, meaning it does not display a file on a
    screen.  Rather, both reading and writing operations happen on a line by line basis only.  Most of the ex editor functionality is still present in Vim as it is quite ubiquitous to text editing.
    You may never use ex commands or you may find them very useful depending on your personal work flow.
  \item \textbf{instance of Vim} This refers to the execution of the program that is Vim.  When you open a terminal and use the Vim command to open Vim or open vim by any other means, you have created
    an instance of Vim.  When you open files and windows and tabs you are still using the same instance of Vim.
  \item \textbf{macros} A macro in Vim simply refers to a key mapping.  This means you can override the function of a keystroke or set of keystrokes by mapping it/them to another set of
    keystrokes. However, you will find that people use this used to mean recording as well.  See definition for recording.
  \item \textbf{mapping} A mapping allows you to change the function of a keystroke or set of keystrokes by mapping it to another set of keystrokes composed of one or more keys.
  \item \textbf{motion} (A.K.A movement or movement command) A motion is any command that moves the cursor.
  \item \textbf{NERDTree} NERDTree is a popular Vim plugin.  It's main feature is a graphical file browser that mimics what you see in many modern IDEs.
  \item \textbf{operator} An operator is a command that operates on text.  Usually this involves changing or deleting the text in some way.  Search operator in the help file for a list of operators.
    As you will find out there are multiple ways to select the text that is being operated on.  Potentially the most powerful way is by typing a motion right after an operator.
  \item \textbf{options} You can think of options as Vim settings.  For a list of all options search for option-list in the help files.
  \item \textbf{Pathogen} Placeholder definition.
  \item \textbf{plugins} Placeholder definition.
  \item \textbf{power user} Placeholder definition.
  \item \textbf{put v paste} Placeholder definition.
  \item \textbf{recording} Placeholder definition.
  \item \textbf{registers} Placeholder definition.
  \item \textbf{scripts} Placeholder definition.
  \item \textbf{session} Placeholder definition.
  \item \textbf{status line} Placeholder definition.
  \item \textbf{swap files} Placeholder definition.
  \item \textbf{tag stacking} Placeholder definition.
  \item \textbf{transparent editing} Placeholder definition.
  \item \textbf{vi} Placeholder definition.
  \item \textbf{vimrc} Placeholder definition.
  \item \textbf{VIMRUNTIME directory} Placeholder definition.
  \item \textbf{Vundle} Placeholder definition.
  \item \textbf{windows} Placeholder definition.
  \item \textbf{word} Placeholder definition.
  \item \textbf{WORD} Placeholder definition.
  \item \textbf{row} Placeholder definition.
  \item \textbf{column} Placeholder definition.
  \item \textbf{buffer-list} Placeholder definition.
  \item \textbf{bottom-line command} (includes searches and anything on bottom line that is not formally a command)
  \item \textbf{buffers} (active, inactive, hidden, unlisted, modifiable, read-only, read errors)
  \item \textbf{commands} and colon commands
  \item \textbf{edit} (as a unit used in undo)
  \item \textbf{jump} and the distinction between and jump and a motion (if it goes into the jump list it is a jump; motions are always within the file; if it goes to another buffer it is a jump; some motions within file are classified as jumps and go in the jump list)
  \item \textbf{modes} (normal, command, ex, insert, visual, select, operator pending, insert normal mode, replace mode)
  \item \textbf{pattern} in context of searching
  \item \textbf{special buffers} Placeholder definition.  (quickfix, help, directory, scratch)
  \item \textbf{tab pages} Placeholder definition.  or tabs
  \item \textbf{text objects} Placeholder definition.  (block and non block text objects)
  \item \textbf{toggle options} Placeholder definition.  (note the set option|nooption|option?|option!|option(ampersand) Syntax)
  \item \textbf{yank} Placeholder definition. not to be confused with copy
  \item \textbf{copy} Placeholder definition. not to be confused with yank
  \item \textbf{modelines} Placeholder definition.  comments that define options automatially p268 Oualline
  \item \textbf{current selection} Placeholder definition.  from user manual page 73
  \item \textbf{real clipboard} Placeholder definition.  from user manual page 73
\end{itemize}

\chapter{Mechanisms of The Vim Universe}
\section{Modes}
Vim is a modal editor.  That simply means that Vim is at all times in only one mode and your keystrokes will mean something different to Vim depending on the modal context in which they are typed.  There
are two main Modes and several others
\section{Commands}
  \subsection{Colon Commands}
  Colon commands, Command-line commands, and Ex Commands are all the same.
  \subsection{Normal Mode Commands}
  \subsection{Other Mode Commands}
\section{Vim Command Grammars}
motions
\section{Concept of an Edit}
\section{Options}
\section{Buffers}
\section{Registers}
\section{Marks}
\section{Tags}
\section{Dotfiles}
\section{Command Line Arguments}
\section{Plugins and Scripts}
\section{Autocommands}
\section{Modelines}
\section{Mouse}
\section{Regex}

\chapter{Capabilities}
Vim is a text editor with an incredible amount of features.  This section will allow you to wrap your mind around the vast set of Vim capabilities.  This section will not dive very deep into how to invoke
functionality.  To do so would be far more information than is necessary to wrap your mind around Vim capabilities.
\section{Navigating a File}
  \subsection{Scrolling}
  \subsection{Jumping}
  \subsection{Search}
  \subsection{Bookmarking}
\section{Editing}
  \subsection{Inserting}
  \subsection{Autocompletion}
  \subsection{Deleting}
  \subsection{Replacing}
  \subsection{Substitutions}
  \subsection{Comments}
  \subsection{Sorting}
  \subsection{Editing Tabular Data}
  \subsection{Formatting}
  \subsection{Cut Copy Paste}
  \subsection{Indenting}
  \subsection{Toggling Case}
  \subsection{Special Characters and Foreign Language}
\section{Work Environment}
  \subsection{Graphical User Interface (GUI)}
  \subsection{File Management}
  \subsection{Undo and Redo}
  \subsection{Mouse Compatability}
\end{document}


