\documentclass[12pt]{book}
\title{Wrap Your Mind Around Vim 8.0}
\author{John Filippone}

\begin{document}
\maketitle

\section{ideas that need to be fit in somewhere}
1. This book does not go into detail about how to use vim, just what it does.
But because vim is so customizable, the how does not matter very much.
Once you know which features you want to use, if it turns out that the way to do those things in vim takes a lot of keystokes and are therefore
not effective, you can remap keys or write scripts to get the same functionality in a totally customized way that works for you
2. section on how to learn and practice effectively.
Use the vim help files and the internet.
Integrate a few commands at a time take it slow at first, think of the most efficient way before you make an edit.

\section{Purpose}
The object of this book is to help new and veteran Vim users efficiently learn the tool by providing a full and concise enumeration of the vast capabilites of Vim 8.0.

\section{Important Historical Perspective}
When Vim was created, it was made by Bram for Bram; meaning, the engineer who developed Vim did so for his personal use.
He developed the features that he personally wanted to use.
As years went on and Bram continued developing, the community asked for other features that Bram did not necessarily care about.
He implemented them anyway as long as he felt enough people would enjoy having them.
After decades of development, Vim has become a vast collection of features.
There are too many for any one person to really need or want or use them all.
The expansive collection of features is meant to accomodate a diverse set of users where each user only needs and/or wants a small subset of all features to achieve their ideal workflow.
It is for this reason that this book focues on efficient learning of Vim rather than complete learning.
Getting the maximum utility from Vim on your particular workflow does not require you to know how to use all the features but rather only 5-10 percent of them.

\section{This Book's Unique Approach}
This is not a 'HOW to use Vim' book.
Instead this book focuses on teaching WHAT Vim can do.
The approach of this book is based on the premise that learning HOW to use a particular feature in Vim is easy; the hard part is figureing out the WHAT feature you want to learn.
Because Vim is such an extensively customizable tool, the HOW of a feature is not particularly important when it comes to achieving efficient workflow.
If the default invocation of a feature does not work for you, there is almost definitely a way to change the invocation to something that does work.
Any how-to elements of this book will be minimal for the sake of brevity.
This book will instead into detail on the WHAT is possible in Vim and WHAT the proper terminology is.
Once you know what the thing you want to do is and what it is called, all it takes some internet searching and some practice.
This is the most efficient approach because it allows you to avoid leaning HOW to use features you don't need or want.
After reading this book you will have effectively wrapped your mind around Vim.
You will have a strong understanding of all the moving parts and which features exist.
Armed with that understanding you can quickly learn the features you need for your particular workflow.
More importantly, the next time your workflow changes, your knowledge of Vim will allow you to quickly assess wether or not Vim is the tool for the job and you will know what to look for.

\section{How Most People Learn Vim}
Most people are thrust into Vi and/or Vim and are forced to frantically learn the basic editing commands.
Even very smart users get stuck in Vim because they do not understand the capabilities of the tool or the technical colloquialisms associated with it.
When struggling with a problem, the new user has two main issues.
They do not know that Vim is capable of doing the thing they want or they know what they want and suspect that Vim can do it but have no idea what the feature is called.
Unfortunately, most books for Vim are recipe books.
They provide tutorial on specific features or describe great ways to solve common problems with Vim.
These are great for picking up tricks but you will brush through hundreds of pages before learning the key set of Vim features that work best for you and the task you are applying them to.
Most will try these 'HOW to' approaches, then quickly give up on them and default to being satified with a cumbersome workflow.
The internet can be very helpful if you know what you are looking for as long as you don't fall into a wormhole of mystifying power-user language.
The weird dialog about buffers and Ex-mode etc. may be easy enough for you to sift through but the real trouble is the former.
When you start using Vim you have no idea what you want to learn!
Most have a vuage idea of what functionality they want, but do not know the proper Vim terminology.
That makes internet searching for a how-to very hard.
Most users will pick up a trick here and there over the years from their coworkers or after getting fed up with a difficult workflow, but never fully wrap their mind around Vim.

\section{How You Ought To Learn Vim}
\begin{itemize}
\item Understand what Vim has to offer. (Wrap Your Mind Around Vim)
\item Pick a subset of key features you need to learn in order to have an efficient workflow for your specific task.
\item Learn each of these features through the Vim User Manual, books, and/or the internet.
\item Practice using them enough to commit them to muscle memory for a truly efficient workflow.
\item When you change your workflow, use your understanding from step one and revert to step 2.
\end{itemize}
This is an efficient method of learning Vim.
Only learn the features you actually care to use.
Know what the tool can do so that later on down the road when you need some other functionality, you know what to look for.
This book aids you with the most important part of this process: Step one.

\section{How to Use This Book}
You can read this straight through or you can skip to the parts you really care about.
It will be valuable to at least skim all of the material so that you come a way with an understanding of the full scope of Vim capabilites.
If you are mystified by the terminology in this book, see the Vim Terminology section in the back.
Key terms will be displayed in bold text throughout the text in order to dispell confusion between words that have a unique meaning when used in a Vim context.

\section{Vim Terminology}
Buffers, word, WORD, motions, modes (normal, command, ex, insert, visual, operator pending), commands, yank v copy, put v paste, registers, text objects, windows, tabs, scripts, plugins, vimrc, dotfiles,
NERDTree, Pathogen, Vundle, mapping, abbreviations, macros, autocommands, options, text object text object is like the w in cw command, one edit, movement command, pattern in context of
searching, bottom-line command (includes searches and anything on bottom line that is not formally a command), the ex editor, toggle options (note the set option|nooption|option? syntax),
set, map, ab, concept of word vs WORD, ctags, tag stacking or tag stacks, power user, transparent editing, session, instance of Vim, buffers (active, inactive, hidden, unlisted, modifiable, read-only,
read errors), special buffers (quickfix, help, directory, scratch), alternate file (saved in the hashtag register) or alternate buffer, VIMRUNTIME dir, events, swp or swap files, status line

\section{Inner workings}
General form of (page 21 O'Reilly Learning the Vi and Vim editors) (command)(number)(text object) or (number)(command)(text object) for change, delete, yank, and put
what Vim puts in the registers (sometimes called buffers?) special registers, black hole register, append to reg, expression reg
switching between modes
concept of a word and WORD p 184 of Oualline
  concept of how vim interprets filenames, comments, identifies, definitions, and printing characters is related
concept of an edit
buffers
tags priority and kind
gvim menu items priority structure
syntax groups and subgroups for coloring (are they used for anything else?)
map overrides normal vim functionality
order of initialization files like vimrc, exrc, etc.
viminfo and retaining session state
ctrl-Q and ctrl-S should be avoided in mappings because they are commands for the terminal to stop and start showing otuput; look into this
+cmd argument; command line argument for vim that starts file at a line number or first instance of a pattern like this: vim +97 file.txt; also works for some vim commands
concept of setting options and maybe an organized overview; sufixes ? and (ampersan) and prefix no;
  options are boolean, numeric, string-related,
    +=, -=, and carrot= for string option values
    can reset all to default with :set all(ampersan)
    set multiple options in one line
Regex support
  character classes [: :]
  collating symbols [. .]
  equivalence classes [= =]
  for substitution, regex works in search portion, some other constructs exist for the replacement portion
  matching beggining and end of words (not a general regex thing? only vim thing?)
  look into isident, iskeyword, isfname, magic, and isprint options
  going to have too look up most update vim 8.0 info on regex support and learn about regex in general a lot so that i can speak the regex language in the book
Basic ways to use features
  gui
  command line args
  options
  ex commands
  mouse
	modelines; comments that define options automatially p268 Oualline

\section{Vim Capabilities}
Editing text
  basic edits
    change
    delete
    insert
    substitute
    append
    yank
    put
    move
    join
      join lines
        joinspaces option will put two spaces in between sentences when joining
    global edits
      deletes, moves, and copies with :g and ex commands
      move blocks of text
        with :g command and pattern matching (helpful if applies to tons of instances in same file hense :g)
    basic edit combos
      transpose characters (xp)
      transpose lines (ddp)
    ex commands (maybe don't need this section)
      usually only useful when need to do something specific everywhere; otherwise there are easier ways to do things
      still useful to at least be aware of the way ex works and can be accessed in vim
      vim allows for extreme flexibility with pattern matching and basic move, copy, delete commands; one must open thier mind to these possiblities when faced with a huge repetative task
      it may be solved with one well-formed, probably long ex command
      nice examples on p84 of OREILLY
    select text with Visual mode to perform an edit on it
      select with any vim cursor navigation technique
      select word, sentence, or paragraph that the cursor is in; ex. 4aw selects 4 words when in Visual mode
      select lines
      slect blocks (can't do this in your average gui editor)
        useful for tables
        insert, append, replace, or change text to multiple lines on the same column
        indent blocks
      repeat a visual selection with gv
      move cursor between beggining and end of selection
      write a sleection to a file
      pipe selection through unix program like sort: select text then use !sort in normal mode (performs operation on lines whether or not you are using visual line mode
      select mode
        behaves more like what 'normal' editors do; backspace to del or type to del and replace
        gh, gH, gctrl-h
        use mouse to do select mode with selectmode option set to mouse?
        can toggle between visual and select mode
    combine edits with motions
  Insertions
    basic insert/append
      start inserting at nifty places
        end of line
        start of line; this can be first non blank char or first column regardless if it is blank
        beggining of new line above or below
        smart indenting
          :set autoindent (next line starts on same indent level on enter)
          indent in insert mode ctrl-t and ctrl-d
          automatically insert leading comment characters when writing a comment that exceeds length of line defined by wrapmargin or textwidth
          audoindent and smartindent
          cindent
            probably best for most programmers
            customization
              cinkeys: keys that trigger a reevaluation of indenting
              cinoptions: customizes a lot of the syntax related indenting behavior; look more into this
              cinwords: words that trigger a reevaluation of indenting (case sensitive regardless of ignorecase option)
          indentexpr
            defines an expression that is evaluated each time a new line is created in a file
            the expression evaluates to a number of indents for that line
            scripts exist in VIMRUNTIME for most languages already; use 'filetype plugin indent on' to use the appropriate one based on filetype
        regular indenting
          use tab in insert mode
          tab over syntax or otherwise defined blocks through various methods
          use command with number arg
          use comparator chars
          use ctrl-g and ctrl-t in insert mode
      insert the char right above or below the cursor with ctrl-y and ctrl-e in insert mode
      insert x amount of something at once i.e. 3iabcESC writes abcabcabc
    completion
      whole line
      current file keywords
      dictionary option keywords
        initially dictionary option is empty; must add a dictionary file or multiple
      thesaurus option keywords
        initially thesaurus option is empty; must add a thesaurus file or multiple
        can search multiple thesauruses
        possible to use a programming thesaurus i.e. retrieve getchar getcwd get direntries getenv getgrent
        can get info in the pop up menu about which thesaurus each suggestion comes from
      current and included file keywords
      tags (ctags)
      filenames
        searches only current directory
      macros
        sleep command can be useful for pausing macro execution
      vim command completion (useful for developing scripts)
      user-defined through completefunc option
      omni-completion
        available for c, css, html, javascript, php, python, ruby, sql, xml, maybe more by now
      spelling suggestions
      generic complete with ctrl-N
        this does potentially all the other complete methods all in one
        the exact complete methods used are defined by the complete option
      where vim searhes for words is managed by complete option
      path option defines where vim looks for files
      abreviations
        create custom abbreviations for writing text with :ab
        abbreviations can be mode specific like maps
        can list abbreviations
        use ctrl-c instead of esc key if you want to exit insert mode without completing abbreviations
        noreabbreviation to avoid chains
      auto generate matching "" '' [] {} () HTML-tags etc.
    special characters
      digraphs
        there are built in shortcuts for typing; just have to learn them
    comments
      easily generate box outlines for box comments with abreviations
      maybe us external program, boxes, to make comment boxes
      auto wrap comments
        this means when you are writing a comment, the editor will automatically input the comment leader on the next line
        does this work for languages other than c?
        managed by formatoptions option
      comment and uncomment blocks easily with block visual mode
    write in foreign languages including right to left languages, farsi and hebrew
      also other languages; maps keyboard to other language characters; need to look into how this is done more
      write right to left
  inserting tabs vs spaces
    expandtab makes the TAB key insert spaces
    override this with CTRL-V before writing a tab (useful for Makefiles)
    combination of tabs and spaces is possible with softtabstop and smarttab and shiftwidth
    control size of tabs
    expandtab does not affect existing tabs but :retab command will convert tabs to whatever vim is currently configured to
    shiftround will make vim always tab to a column that is a multiple of shiftwidth
  substitutions (regex support)
    global
    by line(s)
    confirmation
    context aware substitue
    change and all of its variations: cw, cb, c2b, c0, C, etc
    replace
      type over existing text
  copy/paste
    put from register
      registers hold anything, even edit commands
      registers hold read only metadata like filename
    read command
      !sort filter (:r !sort file-that-needs-sorting)
    copy/paste within a single file
      yank or delete into reg then put from reg
    copy/paste between files
      use ex commands or vi commands to save content to register (or just regular yank) then put (lose undo history when switch between files; use tabs instead)
      read command will copy a full file into the file you are editing
    copy/paste with basic point right lick method
      when smart indenting is on this can result in the progressivly cascading indent problem;
        solve with set paste?  look into this; paste is a shorthand for setting a bunch of other options
        if you use this you can use a pastetoggle key to switch in and out of paste mode quickly
  toggle upper/lower case
  editing with unix commands
    filter text in file through Unix commands
    Run Unix commands from vim
      :!command
        is this technically filter?
        mix this with motion commands
      :sh
      read output of unix command directly into file (:r !command)
  alphanumeric sort lines
  edit binary files -b
  formatting
    justify text so that it looks nice; i.e. put line breaks in nice places automatically
      VIRUNTIME/macros/justify.vim
      gq does it
      gqmotion also works i.e. useful with paragraph motion
      gqip will do it to the paragraph from inside the paragraph instead of having to be on the first line or first char of paragraph
    align text; left right center
      :1,5 right 30
      30 is the width so text will align right to column 30
    use unix program fmt for formating
      set formatprg=fmt
  edit a bunch of files the same way
    batch file i.e. a vim script containing a set of commands that you can run on a file from the unix command line
  encryption
    rot13 encrypt with g??
Cursor Navigation
  motions
    hjkl
    moving vertically through wrapped line
    search and find as a motion
    by word, sentence, paragraph, and section (can set different macros for identifying paragraphs, and sections)
    beggining/end of line
    moving to first non black char of line
    jump/goto
      to line by number (G or goto command)
      to column number with num prefix and |
      goto top, middle, or bottom line on screen (H, M, L) num prefix to go x lines away from top or bottom
      jump to matching or closest ( { [ and more with %
        showmatch will make it so that when you close a matching pair cursor will do a quick jump to matching pair and jump back; kind of annoying if you ask me
        can also jump to next or previous unmatched pair or conditional (if, else); kind of shady, test this out; vim can detect unbalanced matching but not which element is unmatched?
      to the nth percent of a file
      to the nth byte
      to start or end of method
      to next or prev { or }
      to begining and end of comments
    scrolling
      automatic scroll as you type passed bottom
      scroll by screen or half screen
      scroll by line with cursor in place
      scroll relative to cursor (zENTER, z., z-)
      scroll relative to any line (same as z with numeric prefix)
      scroll specified amount specified by scroll option
      control amount of scroll when cursor moving toward the edge of the screen
    return to previous cursor position
    save and goto invisible bookmark called mark
      see all marks with :marks
      special marks exist
      bookmarks are not stored in file; stored in session
      global marks transend files within a vim session; what happens when combinging other commands with a motion to a global mark?
    search as a motion
Searching
  search forward/backward with wrapping or not (regex support)
  fuzzy find
  automatically search word under the cursor
  highlighting of search terms
  quick scrolling through search results
  find
  case sensitive option
  incremental search :set incsearch
  scroll through search history with arrow keys
  search for def of variable (local or global scope)
  jump to macro def
  display macro definition of macro under the cursor
  search for a word under the cursor in the current file and any brought in by include directives: [CTRL-I, ]CTRL-I
  search offset; change locaiton of cursor when finding a search result; i.e. /keyword/b2 will put the cursor on the y of keyword when that keyword is found
GUI
  wrap or no wrap for horizontal lines
    vim might shift lines left and right to make them fit? check this out
    slidescroll option limits such shifting
    sidescrolloff option may also be useful
    listchars option defines characters that act as visual cues for lines that have more content to left or right of the screen
    auto text wrapping (instead of hitting enter)
  left/right scrolling?
  folding
  Color
    syntax highlighting (filetype detection beyond extension name)
      syntax defining files are in VIMRUNTIME/syntax
    column coloring
    colorschemes
      I recommend base16-3024
      customize with colorscheme, hightlight, and background options
      redefine coloring for syntax groups with highlight command
      :highlight will show you all coloring for all syntax groups
      customize syntax group coloring for specific file types by including it in an after script
        ex. for xml put a file called xml.vim in the ~/.vim/after/syntax dir that contains your highlight commands
        also put that dir in runtime path like this :set runtimepath+=~/.vim/after/syntax
      create your own syntax file
        possibly for a custom file extension; in this case you can get vim to automatically detect your extension and use your syntax
    color test
  highlighting
    specific column(s)
    search terms
    matching { [ ( etc.
  line numbering
    normal numbers
    relative numbers
  redraw screen Crl-l
  mode indicator
  show white space chars with :list
  gvim
    basic point click and scroll you expect
    gvimrc
    checkout scrollbars; customizing them; use of scroll bars with multiple windows
    menus
      control order of root menus at the top (vim has a menu priority system including default menu priorities for existing menus that might be important to mention)
      control order of items within menus
      control spacing between menu items
      make your own menus and menu items that execute vim commands
      tear off menues
      make menues dependent on mode
      special menu names "ToolBar" and "PopUp"
      noremap menu mappings
    toolbar
      treated similarly as a one dimentional menu
      uses bmp images as icons
      define tooltips for icons
    Windows gvim
      self installing executable
      basic copy/pasting is compatible with System clipboard
    look into gvim specific options
    title bar string
      make title bar say the name of the file being edited
      also can manually set title bar
      adjust max length of title bar string
      can also manaage string in minimized window bar
    customize mouse behavior
      turn mouse on and of by mode
      mouse mappings
      double click time
      turn off mouse pointer when typing
    make select mode default over visual mode or choose toggle method
    invoke certain dialog boxes with commands instead of menu items
    confirmation dialog boxes when action will delete data
    browse options; badass; must write about this; maybe display the browse options menu in the book; allows you to browse an organized list of options, see defs and toggle; :browse set
    set font with guifont; set guifont=* will bring up a menu of fonts to choose from
    change style of the cursor
  convert files to html with :TOhtml command
  vimdiff gui
    execute vimdiff from commandline
    non unix vim versions come with a vim version of diff (unix of course uses the built in command)
    diffexpr option defines replacement expression
  showcmd
  vim shows messages on the command line after certain commands like write, you can make these short with shortmess option
  errorbells and visualbells
  format status line
    customize reporting changes with report
  list mode
    customize chars list mode uses with listchars
Workflow
  ga will show you ascii number or char under cursor
  Backwards compatible with vi
  Mouse compatability
    behave option can help make mouse behavior more like the way you are use to
  Environment
    windows (splits)
      from initialization or during session
      add/remove/switch windows
      displaying two instances of same file
      multi window ex commands
      move cursor from window to window
      resize windows
        size is in lines and columns; change to n value or +/- lines/columns
        options
          when you switch to a window it will auto resize to your winheight and winwidth option values
          equalalways, cmdheight, winminwith, winminheight
      move windows around display
        swap windows with retaining size or retaining window layout
        rotate windows of a column or row
      tabs
        can vim open silent tabs?
        tabnew, tabclose, tabonly, tabprev, tabnext, CTRL page up/down
        cycle from last tab straight to first and vis versa
      split search; same as * command except does it in a new split window; ctrl-w ctrl-i
      status lines for each window is a toggle
      :[n]split [++opt] [+cmd] filename
      equalsalways option to always make new windows equal size
      look into :new and the autocommands it executes
      :sview and :sfind; and generally add s to a lot of commands to get them to do the same thing in a split window
      look into conditional split commands p179 OREILLY
      execute command on all windows with :windo
      open tag or file (on path defined in option variable, path) in new window or tab
      close windows
        quit, close, hide, and close all others
          hidden option
      windows can have their own status line or not
      bind the scrolling of two windows vertical and/or horizontal with scrollbind
        look into syncbind as well
      preview window
    customize vim stuff with env vars
    edit multiple files
      vim f1.txt f2.txt ...
      see files with :args
      use next and previous to shuffle through or shortcut to alternate file
    move around to different files with buffer switching commands
    save current session settings to vimrc file with :mkvimrc filename
  home dir dot file executed first then the one in cwd; possiblity for different settings in different envirionments
  many ways to get vim to display current vim state (options, abreviaitons, registers, etc.)
  ctags comatability
    :tag command (look further into this)
    :tags shows list of tags that you have traversed through
    :tagselect shows instances of the same tag
    tag stacking
      related to jump stack
    tags option defines where vim looks for tags
    compatible with etags
  copy to system clipboard from Vim
    system clipboard register is "* ?  verify this
    in gvim this is default I think? verify!
  edit-compile-edit cycle with quick fix
    find out if this can be done in many languages, not just c
    compile from vim with :make filename
    define program used to make with makeprg option
    jump to location of errors
    easily move between locations of all the errors
  rename refactor with vimgrep and quick fix
    p 283 O'REILLY
  automatically jump to compile error lines
    look into redir command
  File management
    creating files
    opening/closing files
    opening files read only
    open file with cursor at position
      on line number
      on first occurance of pattern
      on last line
    recovering edited buffers after crash
      swap files
      vim will try not to override swp files; it will name the next one swo then swn and so on
      control when swp file gets written; 4 seconds or 200 chars is default
      control where swp files are written; can be multiple files
      preserve command will write to swp
    write to open file or any file by name or create new file and write to it
    write or append part of a file to a new file
    open a bunch of files in one session and switch between them on one screen
    open file reguardless of file type
      edit binary files; set the binary option for safer editing; this turns on/off some vim settings that make editing binaries safer but editing binaries is still not recommended
    file browsing
      delete and rename files and directories
      search thorugh dir like a normal vim file
      open file in split window
      can you do a quiet open? where file goes into a new tab?
      edit file that is under cursor with gf which is same as :find filename
        path option defines where vim looks for filenames
      wildcard menu for comandline filename completion
        siffixes option defines extensions that are given low priority in wildcard menu
        look into wildmode
    directly edit compressed files and directories (transparent editing)
    configure vim to produce backup files
      automatically backup files
      specify location for backups with backupdir
      backup, writebackup, backupcopy, backupdir, backupnext options
      vim will add a ~ extension to backup files; manage the extension with backupext
      patchmode?
      difference between backup and write backup?
      if you are in a write protected file and you don't want to loose changes just write to another file from vim with :write filename
    teach vim new filetypes based on unfamiliar file extensions (ex. let vim know that a .inc extension is a C file)
    fileformats
      this has to do with file formats with regards to end of line chars; unix is line feed, apple is return and dos is both
      fileformats option manages this
      vim auto detects unix,apple,dos file formats regarding the end of line chars and is able to read all formats
      save a file in a different file format
    control wether or not files end with and EOL char with endofline option
    encryption
      key option holds your encription key
      encrypt an existing file or new file
      vim encryption is weak
    when a file is changed by another program while you are editing and then you write, vim will give you a warning as ask if you want to continue
    remove white space at ends of lines with substitude command; suggestion is to rig an auto command to do this on write
    cd and pwd command mode commands
  command tab completion :e :h etc.
  vim diff
  Undo/redo
    multi level undo/redo (aka infinite undo?)
      by default undolevels is 1000
    undo by line
    repeat an edit (this is not redo; it is repeat edit)
    repeat an insert with ctrl-A
    special tricks for repeating substitutions p80 of OREILLY
    undo/redo branching; see usr32.txt
  Folding
    nested folds
    fold types
      syntax based
        use set foldmethod=syntax and set foldenable
      indent based
        use set foldmethod=indent
        use set foldlevel=n to define how much you see and toggle foldlevel with zm and zr
      regex defined
      manually defined
        use any motion commands with a z command
      marker defined
      diff differences are folded
    operate folds recursively or one at a time (open, close, delete, toggle) (is toggle same as open/close?)
    look into foldenable an dfoldlevel
    foldcolumns visual que on the left if you're into that
    execute command on a folded line will execute that command on every line in the fold
    keep manually defined fold with :mkview and :loadview
  edit files remotely
    scp, ftp, sftp, http, dav, rcp
  edit files that contian a given word
    vim `grep -l 'special-term' *`
  save session state
    viminfo file
      managed by the viminfo option parameters
        save lines for each register up to n lines
        number of search pattern items to save
        number of command line commands to save
        max files vim will maintain info on
      contains
        command line history
        search string history
        input line history
        registers
        file marks
        last search and substitute patterns
        buffer list
        global variables
    mksession
      :mksession filename will save just about everything about the session into a file that can be sourced in a vim session later
      managed by sessionoptions option
      what you can save
        empty windows
        windows and tabs
        hidden and unloaded buffers
        current dir
        manually created folds, state of folds, local fold options
        global vars
        help window
        local options i.e. options set locally to a particular window or buffer
        options
        window sizes
        dir in which session file is located
        unix end of line format
        window position on the screen
  quick command history reference with ctrl-f in commandline; full history with :history command; can set amount of remembered commands
  configure vim to execute python, perl, tct, and probably more inside of vim
  can make insert mode the default
Customization
  if it is a thing, you can probably customize it
  conifgure backspace key with backspace option
  Vim configurations when compiling Vim
  Vim script
    most common commands have one or two letter shortcuts; when using these in script makes them hard to read; instead you can use explicit commands like :copy and :delete for just about anything
    .vim scripts in vim runtime directories serve as good examples
    if/else
      inline, elseif and else are optional
    while
    datetime support
    variables
      optionally explicitly define variable scope
      handle strings and ints
      literal string with single quote or simple string with double; simple expands escape characters, literal string does not
      special variables for env vars, registers, etc
      built in vim variables
      ints can be decimal, octal, or hex, (or binary?)
      can delete variables
    comparisons
      can do basic boolean comparisons on ints and strings
      can check if string matches regex with comparison; string =~ regex or the compliment: string !~ regex
      ignorecase string comparision
        to ignore add ? (i.e. ==?); must match case: add hashtag
        default string compare honors ignorecase option
    enter filenames dynamically
      file name under cursor, curretn file, alt file, lots more
      modify that filename by adding/removing abs path, get extension etc.
      to experiment:
        put cursor over the name of a file
        :echo expand("<cfile>:p")
        this will echo the abs path of the file under the cursor
        this could be pretty powerful..look more into applications of this
    concatinate with .
    built in functions
      look into help file usr41.txt(has underscore after usr but tex hates it) for details on over 200 built in functions;
      execute
      echo
        can also echo in color
      filetype detect
      exists
      confirm; dialog boxes wow cool such gui
    define functions
      with/without args
      call
      return
      if you define a function that already exists you will get an error but you can override it with function! FuncNameThatAlreadyExists
      range functions?
      function FuncName() abort will define a func that aborts on fist error
      function FuncName(start, ...) means it reqires one arg and can have up to 20 more stored in a:1, a:2, etc; a:0 is the count of extra args
      list user def functions with :function
      can delete functions
    defind command mode commands
      kind of like mapping for command mode; must start with capital letter
      list user def commands with :command
      can clear user def commands
      can take arguments
        predetermined number
        0 or 1
        any number
        one or more
        range; look in to what it means to have a range argument???
    get function to call periodically
      statusline trick p.202 of O'REILLY
      better way?
    arrays
    dictionaries
    context conversion for variables (most notably string to int..verify this!)
    execute function based on filetype (autocommand)
    plugins
    basic ex scripting called from bash or :so
      possible to do complicated things if you know what you're doing
        sorting glosseries
        updating last mod time of the file
        changing tabwidth for different file types
      can use comments
      use spaces in ex commands to make them more readable
    execute script based on time of day
    everything you put in vimrc is technically vim script
    Autocommands
      execute arbitrary script on event happens
      decide which files will be subject to the autocmd script
      groups
      delete commands when they are no longer needed
        allows for dynamic changing of autocommands and which files they execute on
      can delete groups but make sure to delete all commands in group first
      autocommands are by default not triggered by other autocommands but you can specify that they can be if you want
      can choose to ignore events
    evaluating expressions (not only int math?)
      add, mult, sub, int div, mod, negation
    increment and decrement integers even in octal and hex form
      nrformats manages which number formats are recognized
  Remapping keys
    map commands affect keys in one or more modes; operator pending mode comes in to play here with omap
    remap function keys that are mapped by the terminal (p.110 OREILLY) (look into this more)
      looks deeper into which exact keys can be remapped; look into the ability to map alt key, p344 Oualline says it can be done on gui
    chains of maps are possible but also preventable with noremap
      can also set noremap option to make noremap the default behavior; might break some scripts!
    unmap; takes only one arg and sets that arg to default functionality
    remove all mappings with mapclear
    list mappings with :map
  command combinations like ddp
  Macros (aka @-Functions)
  pre/post edit routines
    VIMINIT env var
    .vimrc for initialization state (and gvimrc; executed after vimrc if using gvim)
    .viminfo for session context
    can define scripts that run when exiting buffers
    pre and post scripts run automatically when switching buffers in the same session
  customize how vim defines constructs
    sentence, paragraph, word, WORD, comment, filename, identifiers, keywords, matchingpairs, definitions, includes, errorformat
  customize how indent events are triggered
    cinwords, cinkeys, cinoptions
  customize :make command
  choose binary or linier search for grep
  verbose option so you can see what vim is doing under the hood
    there are clearly defined levels for this Oualline 402
Applied Vim to specific languages
  HTML
  Python
Support
  Vim help files with tags
  vimtutor
  use K to open man page for word under cursor
    K takes a prefix number i.e. 2K to specify section number in the man page
    keywordprog option defines progrm that is executed by K i.e. man in unix
    keywordprog can be modified for example to nothing, signaling vim to use :help
    the iskeyword option defines what K defines as a word
    using vim to browse man pages is lame unless you run it through ul -i like this: man date | ul -i | vim; verify this
      alternatively use vim substitute to fix; maybe rig it up in a autocmd; Oualline p 167

Vim resources I need to check out and maybe provide
  vi Lover's Home Page by Thomer M. Gil at http://www.thomer.com/vi/vi.html
  Vi Pages by Sven Guckes at http://www.vi-editor.org
  appendix A of O'REILLY
    contians important info about command syntax structure
    contains documentation of ex commands
  appendix B of O'REILLY
    contians documentation of some "most important" options
  tutorial from unix world magazine by walter zintz
    old url: http://www.networkcomputing.com/unixworld/tutorial/009/009.html

\end{document}


